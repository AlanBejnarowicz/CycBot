\section{Results} % (~40%)
% The results section of your report should contain the actual results. No interpretations, no extrapolations, just the (processed) data. Displayed nicely with confidence limits included in some form - error bars, standard deviations, whatever. The description of how you processed your data should be included in either the methods or else the results section. Note that unexpected and outlying results should also be included, especially if they are repeatable. In theory, negative results are just as important as positive results. It is just as scientifically valid and useful to state that this approach to this problem doesn't work, as it is to find an approach that does work. As the project focuses on nonverbal communication in interactions with humans, you should include results of one or more human-robot interaction (HRI) experiments evaluating this communication and demographic data for participants (age, gender, prior robot interaction experience).

\subsection{HRI experiment data}

\par Participants interacted with the~robot while it expressed different emotions in a~cycle shown in Figure~\ref{fig:cycbot-state}. Participants could touch the~robot's head or belly freely during the~interaction. We recorded number of touches during each robot behaviour. The~collected data is summarized in Table~\ref{tab:table-HRI-data}.

\begin{table}[!ht]
    \caption{Count of touches during each robot behaviour in HRI experiment.}\label{tab:table-HRI-data}
    \begin{center}
        \begin{tabular}{ccccc}
            \hline
            \multicolumn{1}{c}{Expression} & Touch count \\
            \hline
            Idle & --- \\
            Interested & --- \\
            Happy & ---  \\
            Annoyed & --- \\
            Sad & --- \\
            \hline
        \end{tabular}
    \end{center}
\end{table}

\subsection{Fabrication}

\begin{figure}[h]
    \centering
    \includegraphics[width=0.4\textwidth]{data/cyc_bot_section.png}
    \caption{CycBot section view showing internal components and silicone parts.}
    \label{fig:cycbot-section}
\end{figure}

Hardest part of CycBot to make was the silicone belly. To make the silicone belly we designed and 3D printed
a mold shown in Figure~\ref{fig:forma-cyca}. The mold consists of four parts:
split outer shell, inner core, and a cap to hold the inner core in place. The first step in making
the silicone belly was to assemble the mold by placing the inner core inside the outer shells and securing it with the cap.
Then we sealed all edges with duct tape to prevent silicone leakage. 
Because of the wall thickness of the belly (3 mm on the walls and 8 mm on the bottom) and the mold 
being relatively tall we were concerned about air bubbles getting trapped inside the silicone during pouring,
which would make our part faulty.
To keep our silicone bubble free we put our silicone mixture in a vacuum chamber for 15 minutes and 
then poured it slowly into the mold through a small hole on the top of the mold cap. After pouring
we placed the mold in a vacuum pot for 45 minutes to further eliminate any remaining air bubbles.

Fortunately our part came out bubble free on the first try. 
We then placed the mold in an oven at 60°C for 2 hours to cure the silicone. We put our mold in the oven
for longer than recommended by the silicone manufacturer to make sure the silicone was fully cured because 
3D printed molds tend to absorb some heat during the curing process.

After curing we disassembled the mold and removed the silicone belly. The final step was to trim any excess
silicone and glue the belly to the robot frame using Sil-Poxy silicone adhesive. Figure~\ref{fig:cycbot-section}
shows a section view of the CycBot with all internal components and silicone parts.

In the end our mold design worked very well and we were able to make a high quality silicone belly
for our robot.

\begin{figure}[h]
    \centering
    \includegraphics[width=0.3\textwidth]{data/forma_cyca.png}
    \caption{3D model of mold used to cast CycBot's silicone belly.}
    \label{fig:forma-cyca}
\end{figure}

\subsection{HRI experiment}

Signals used in experiments are robot behaviours explained in the table above \ref{table-emotions}

\textbf{Hypothesis H1:} Participants will touch robot more while presenting Interested or Annoyed expression.

\textbf{Measure:} Number of touches during each behaviour.
\textbf{Analysis:} One-way ANOVA test to compare means of touches during each behaviour.
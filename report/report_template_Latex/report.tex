%%%%%%%%%%%%%%%%%%%%%%%%%%%%%%%%%%%%%%%%%%%%%%%%%%%%%%%%%%%%%%%%%%%%%%%%%%%%%%%%
%2345678901234567890123456789012345678901234567890123456789012345678901234567890
%        1         2         3         4         5         6         7         8

\documentclass[letterpaper, 10 pt, conference]{ieeeconf}  % Comment this line out if you need a4paper

%\documentclass[a4paper, 10pt, conference]{ieeeconf}      % Use this line for a4 paper

\IEEEoverridecommandlockouts                              % This command is only needed if 
                                                          % you want to use the \thanks command

\overrideIEEEmargins                                      % Needed to meet printer requirements.

%In case you encounter the following error:
%Error 1010 The PDF file may be corrupt (unable to open PDF file) OR
%Error 1000 An error occurred while parsing a contents stream. Unable to analyze the PDF file.
%This is a known problem with pdfLaTeX conversion filter. The file cannot be opened with acrobat reader
%Please use one of the alternatives below to circumvent this error by uncommenting one or the other
%\pdfobjcompresslevel=0
%\pdfminorversion=4

% See the \addtolength command later in the file to balance the column lengths
% on the last page of the document

% The following packages can be found on http:\\www.ctan.org
\usepackage{graphics} % for pdf, bitmapped graphics files
%\usepackage{epsfig} % for postscript graphics files
%\usepackage{mathptmx} % assumes new font selection scheme installed
%\usepackage{times} % assumes new font selection scheme installed
%\usepackage{amsmath} % assumes amsmath package installed
%\usepackage{amssymb}  % assumes amsmath package installed
\usepackage{graphicx}
\title{\LARGE \bf
Title of your project*
}


\author{Name of students$^{**}$ % <-this % stops a space
\thanks{*The contents of this document are adopted from “Writing Good Scientific Papers” by Bridget Hallam and put into the IEEE conference paper format.}% <-this % stops a space
\thanks{$^{**}$Students' Information (student ID, study program, email address)}%
}


\begin{document}



\maketitle
\thispagestyle{empty}
\pagestyle{empty}


%%%%%%%%%%%%%%%%%%%%%%%%%%%%%%%%%%%%%%%%%%%%%%%%%%%%%%%%%%%%%%%%%%%%%%%%%%%%%%%%
\begin{abstract}

This document provides general information on how to prepare the report for your exam project as a scientific paper. Choose a precise and attractive title that reflects your project. Include the name and information of all team members (names in alphabetical order) For the abstract, write 1-2 sentences about what you did, 1-2 about your results, and 1 with the main conclusion/take-home message (max 200 words). Follow the following structure for your report with suggested length for each section. Limit your report to 6 pages including figures, tables, references, etc. An example paper is uploaded to the ItsLearning platform (Klausen et al “Signalling Emotions with a Breathing Soft Robot”).

\end{abstract}


%%%%%%%%%%%%%%%%%%%%%%%%%%%%%%%%%%%%%%%%%%%%%%%%%%%%%%%%%%%%%%%%%%%%%%%%%%%%%%%%
\section{Introduction ($\sim\% 10$)}

The introduction to your report is important. It sets the scene, showing the state of research in the tiny area relevant to your work. Make sure that you know and write down which research question you are addressing, how other people have succeeded in getting some way towards an answer, and how you are trying to potentially get further forward or replicate their results. Read the papers that you cite, to check that they say what you think. 

\section{Methods ($\sim\% 25$)}
It is always a good idea to write down your methods in detail, with all parameters written down, in a permanent place (e.g., a hard-backed book), while you are still doing the work, and to label all results with which experiment they belong to. Of course, if you will be working on computer, take back-ups seriously. Ideally, your report should contain enough information about the methods used to enable another student/researcher to replicate your experiments. The methods section should concisely describe your design as well as its materials and fabrication. Also specify which control strategy you are using for your robot and in case you are using a bioinspired design or another inspiration, state which animal(s)/character(s) it is inspired by and how. Lastly, describe the methodology of your human-robot interaction (HRI) experiment (procedure, measures/questions posed etc.).

\section{Results ($\sim\% 40$)}
The results section of your report should contain the actual results. No interpretations, no extrapolations, just the (processed) data. Displayed nicely with confidence limits included in some form — error bars, standard deviations, whatever. The description of how you processed your data should be included in either the methods or else the results section. Note that unexpected and outlying results should also be included, especially if they are repeatable. In theory, negative results are just as important as positive results. It is just as scientifically valid and useful to state that this approach to this problem doesn't work, as it is to find an approach that does work. As the project focuses on nonverbal communication in interactions with humans, you should include results of one or more human-robot interaction (HRI) experiments evaluating this communication and demographic data for participants (age, gender, prior robot interaction experience).

\section{Discussion ($\sim\% 20$)}
The discussion section is where you put your interpretations, extrapolations etc. It normally starts with concrete conclusions that come directly from your data itself, then gradually becomes more abstract as you extrapolate into the future and into broader research areas. You should start concrete, to show people that you are working with fact and not fantasy. But there is no harm in claiming potential greater significance for your results, if justifiable. Go back and check again, that what you have written in your discussion section ties in properly with the results you included in your report. What is the main message that you want readers to take home? Make this the primary focus of your discussion! The rest of the discussion should provide supporting arguments for this, your main thread. Your discussion section could address, e.g., select aspects of the robot’s performance compared to the state-of-the-art and results and implications of your HRI experiment(s).

\begin{table}[h]
\caption{Typical Silicone-based soft materials}
\label{table_example}
\begin{center}
\begin{tabular}{ccccc}
\hline
&
\multicolumn{2}{c}{Ecoflex} 
& Elastosil & Sylgard\\
\hline
Type & 00-30 & 00-50 & M4601 & 184\\
Shore hardness & 00-30 & 00-50 & 28A & 50A\\
Elongation at break & 900\% & 980\% & 700\% & 150\%\\
\hline
\end{tabular}
\end{center}
\end{table}


\begin{figure}[h]
\centering
\includegraphics[width=0.6\linewidth]{figure_1.jpg}
\caption{A soft pneumatic elephant trunk-inspired manipulator~\cite{Wilson2007}}
\label{figure1}
\end{figure}
   


\section{Conclusion ($\sim\% 5$)}

A conclusion may review the main points of the work, do not replicate the abstract as the conclusion. A conclusion might elaborate on the importance of the work or suggest applications, extensions, and improvements. 

\addtolength{\textheight}{-12cm}   % This command serves to balance the column lengths
 
%%%%%%%%%%%%%%%%%%%%%%%%%
\section*{Appendix}

Insert the link to an edited video (unlisted Youtube) of your project and a brief description of supplementary materials (code, properly labelled data, etc.). State each student’s contribution to the project and report according to the provided template on ItsLearning.



%%%%%%%%%%%%%%%%%%%%%%%%%



\begin{thebibliography}{99}

\bibitem{Wilson2007} J. F. Wilson, I. Norio, Bellows-type springs for robotics, Proc. Adv. Spring Technol. JSSE 60th Anniversary Int. Symp., 109–119 (2007).
 






\end{thebibliography}




\end{document}

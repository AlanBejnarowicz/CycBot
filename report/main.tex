% \documentclass[letterpaper, 10pt, conference]{ieeeconf}  % Comment this line out if you need a4paper

\documentclass[a4paper, 10pt, conference]{ieeeconf}      % Use this line for a4 paper

\IEEEoverridecommandlockouts                              % This command is only needed if 
                                                          % you want to use the \thanks command

\overrideIEEEmargins                                      % Needed to meet printer requirements.


% The following packages can be found on http:\\www.ctan.org
% \usepackage{graphics} % for pdf, bitmapped graphics files
%\usepackage{epsfig} % for postscript graphics files
%\usepackage{mathptmx} % assumes new font selection scheme installed
%\usepackage{times} % assumes new font selection scheme installed
%\usepackage{amsmath} % assumes amsmath package installed
%\usepackage{amssymb}  % assumes amsmath package installed
\usepackage{graphicx}

\usepackage{hyperref}
\hypersetup{
    colorlinks=true,
    linkcolor=black,
    filecolor=magenta,
    urlcolor=blue,
    citecolor=blue,
    pdfpagemode=UseOutlines
}
\urlstyle{same}

\usepackage[backend=biber,style=ieee]{biblatex}
\addbibresource{references.bib}


\title{\LARGE \bf{CycBot: A~Cyclical Soft Robot for~Non-verbal Communication with~Humans}}

\begin{document}

\author{Alan Bejnarowicz, Jan Górski, Maciej Piszczek}


\maketitle
\thispagestyle{empty}
\pagestyle{empty}


%%%%%%%%%%%%%%%%%%%%%%%%%%%%%%%%%%%%%%%%%%%%%%%%%%%%%%%%%%%%%%%%%%%%%%%%%%%%%%%%
\begin{abstract}

In this project, we present CycBot --- a soft robot designed for non-verbal communication with humans. The robot's design incorporates silicone materials and 3D printing. Our robot was inspired by toys like Furbies and Tamagotchis~\cite{furby-wiki,tamagotchi-wiki}, aiming to create an engaging interaction experience. We achieved a successful design and fabrication of CycBot, demonstrating its ability to express emotions using soft robotics actuator. In our Human-Robot Interaction (HRI) experiment, we evaluated the robot's effectiveness in conveying emotions and engaging users.

\end{abstract}


%%%%%%%%%%%%%%%%%%%%%%%%%%%%%%%%%%%%%%%%%%%%%%%%%%%%%%%%%%%%%%%%%%%%%%%%%%%%%%%%
\section{Introduction} % (~10%)
% The introduction to your report is important. It sets the scene, showing the state of research in the tiny area relevant to your work. Make sure that you know and write down which research question you are addressing, how other people have succeeded in getting some way towards an answer, and how you are trying to potentially get further forward or replicate their results. Read the papers that you cite, to check that they say what you think.

\subsection{Design of the~robot}

\par Our robot, CycBot, was designed in Fusion~360 and fabricated using a~combination of 3D~printing and silicone molding. The~robot's body consists of a~3D-printed frame, while the~body and ears are made from Ecoflex~00-30 silicone. The~robot uses one servo motor to rotate head, three air pumps, and two air valves to control the~inflation and deflation of the~ears. To make the~robot more human-like, we cast `belly' and ears using skin-coloured silicone. To keep the~robot upright we poured small amount of hard silicone on~the~bottom of its belly. The~robot is controlled using the~Arduino~Uno microcontroller~\cite{arduino-uno}, which manages the~servo motor and air pumps/valves. The robot's behaviour is programmed to respond to human interactions, allowing it to express different emotions through ear movements and head rotations.

\par We designed our robot to show five distinct emotions: interested, happiness, annoyed and sadness. Each emotion is expressed through a~combination of ear movements and head rotations, based on the~human input. The~inflatable belly of the robot has a~pressure sensor that detects when a~human touches it and for how long.


\subsection{Fabrication}

\begin{figure}[h]
    \centering
    \includegraphics[width=0.4\textwidth]{data/cyc_bot_section.png}
    \caption{CycBot section view showing internal components and silicone parts.}
    \label{fig:cycbot-section}
\end{figure}

Hardest part of CycBot to make was the silicone belly. To make the silicone belly we designed and 3D printed
a mold shown in Figure~\ref{fig:forma-cyca}. The mold consists of four parts:
split outer shell, inner core, and a cap to hold the inner core in place. The first step in making
the silicone belly was to assemble the mold by placing the inner core inside the outer shells and securing it with the cap.
Then we sealed all edges with duct tape to prevent silicone leakage. 
Because of the wall thickness of the belly (3 mm on the walls and 8 mm on the bottom) and the mold 
being relatively tall we were concerned about air bubbles getting trapped inside the silicone during pouring,
which would make our part faulty.
To keep our silicone bubble free we put our silicone mixture in a vacuum chamber for 15 minutes and 
then poured it slowly into the mold through a small hole on the top of the mold cap. After pouring
we placed the mold in a vacuum pot for 45 minutes to further eliminate any remaining air bubbles.

Fortunately our part came out bubble free on the first try. 
We then placed the mold in an oven at 60°C for 2 hours to cure the silicone. We put our mold in the oven
for longer than recommended by the silicone manufacturer to make sure the silicone was fully cured because 
3D printed molds tend to absorb some heat during the curing process.

After curing we disassembled the mold and removed the silicone belly. The final step was to trim any excess
silicone and glue the belly to the robot frame using Sil-Poxy silicone adhesive. Figure~\ref{fig:cycbot-section}
shows a section view of the CycBot with all internal components and silicone parts.

In the end our mold design worked very well and we were able to make a high quality silicone belly
for our robot.

\begin{figure}[h]
    \centering
    \includegraphics[width=0.3\textwidth]{data/forma_cyca.png}
    \caption{3D model of mold used to cast CycBot's silicone belly.}
    \label{fig:forma-cyca}
\end{figure}

\subsection{HRI experiment}

Signals used in experiments are robot behaviours explained in the table above \ref{table-emotions}

\textbf{Hypothesis H1:} Participants will touch robot more while presenting Interested or Annoyed expression.

\textbf{Measure:} Number of touches during each behaviour.
\textbf{Analysis:} One-way ANOVA test to compare means of touches during each behaviour.

\section{Results} % (~40%)
% The results section of your report should contain the actual results. No interpretations, no extrapolations, just the (processed) data. Displayed nicely with confidence limits included in some form - error bars, standard deviations, whatever. The description of how you processed your data should be included in either the methods or else the results section. Note that unexpected and outlying results should also be included, especially if they are repeatable. In theory, negative results are just as important as positive results. It is just as scientifically valid and useful to state that this approach to this problem doesn't work, as it is to find an approach that does work. As the project focuses on nonverbal communication in interactions with humans, you should include results of one or more human-robot interaction (HRI) experiments evaluating this communication and demographic data for participants (age, gender, prior robot interaction experience).

\subsection{HRI experiment data}

\par Participants interacted with the~robot while it expressed different emotions in a~cycle shown in Figure~\ref{fig:cycbot-state}. Participants could touch the~robot's head or belly freely during the~interaction. We recorded number of touches during each robot behaviour. The~collected data is summarized in Table~\ref{tab:table-HRI-data}.

\begin{table}[!ht]
    \caption{Count of touches during each robot behaviour in HRI experiment.}\label{tab:table-HRI-data}
    \begin{center}
        \begin{tabular}{ccccc}
            \hline
            \multicolumn{1}{c}{Expression} & Touch count \\
            \hline
            Idle & --- \\
            Interested & --- \\
            Happy & ---  \\
            Annoyed & --- \\
            Sad & --- \\
            \hline
        \end{tabular}
    \end{center}
\end{table}


\section{Discussion} % ~20%
The discussion section is where you put your interpretations, extrapolations etc. It normally starts with concrete conclusions that come directly from your data itself, then gradually becomes more abstract as you extrapolate into the future and into broader research areas. You should start concrete, to show people that you are working with fact and not fantasy. But there is no harm in claiming potential greater significance for your results, if justifiable. Go back and check again, that what you have written in your discussion section ties in properly with the results you included in your report. What is the main message that you want readers to take home? Make this the primary focus of your discussion! The rest of the discussion should provide supporting arguments for this, your main thread. Your discussion section could address, e.g., select aspects of the robot's performance compared to the state-of-the-art and results and implications of your HRI experiment(s).

\section{Conclusion} % ~5%

A conclusion may review the main points of the work, do not replicate the abstract as the conclusion. A conclusion might elaborate on the importance of the work or suggest applications, extensions, and improvements.

\addtolength{\textheight}{-12cm}   % This command serves to balance the column lengths

%%%%%%%%%%%%%%%%%%%%%%%%%
\section*{Appendix}

Insert the link to an edited video (unlisted YouTube) of your project and a brief description of supplementary materials (code, properly labelled data, etc.). State each student's contribution to the project and report according to the provided template on ItsLearning.


\printbibliography[title={References}]


\end{document}

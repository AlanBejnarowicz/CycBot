\documentclass[letterpaper, 10 pt, conference]{ieeeconf}  % Comment this line out if you need a4paper

%\documentclass[a4paper, 10pt, conference]{ieeeconf}      % Use this line for a4 paper

\IEEEoverridecommandlockouts                              % This command is only needed if 
                                                          % you want to use the \thanks command

\overrideIEEEmargins                                      % Needed to meet printer requirements.


% The following packages can be found on http:\\www.ctan.org
\usepackage{graphics} % for pdf, bitmapped graphics files
%\usepackage{epsfig} % for postscript graphics files
%\usepackage{mathptmx} % assumes new font selection scheme installed
%\usepackage{times} % assumes new font selection scheme installed
%\usepackage{amsmath} % assumes amsmath package installed
%\usepackage{amssymb}  % assumes amsmath package installed
\usepackage{graphicx}
\title{\LARGE \bf
CycBot: A Cyclical Soft Robot for Nonverbal Communication with Humans
}

\begin{document}

\author{Alan Bejnarowicz, Jan Górski, Maciej Piszczek }


\maketitle
\thispagestyle{empty}
\pagestyle{empty}


%%%%%%%%%%%%%%%%%%%%%%%%%%%%%%%%%%%%%%%%%%%%%%%%%%%%%%%%%%%%%%%%%%%%%%%%%%%%%%%%
\begin{abstract}

In this project, we present CycBot – a soft robot designed for nonverbal communication with humans. 
The robot's design incorporates silicone materials and 3D printing. Our robot was inspired by 
toys like Furbies and Tamagotchis, aiming to create an engaging interaction experience.
We achieved a successful design and fabrication of CycBot, demonstrating its ability to express
emotions using soft robotics actuator.

\end{abstract}


%%%%%%%%%%%%%%%%%%%%%%%%%%%%%%%%%%%%%%%%%%%%%%%%%%%%%%%%%%%%%%%%%%%%%%%%%%%%%%%%
\section{Introduction ($\sim\% 10$)}

Our robot, CycBot, was designed in Fusion 360 and fabricated using a combination of 3D printing
 and silicone molding. The robot's body consists of a 3D-printed frame, while the body and ears 
 are made from Ecoflex 00-30 silicone. The robot uses one servo motor to rotate head, three 
 air pumps, and 

\section{Methods ($\sim\% 25$)}
It is always a good idea to write down your methods in detail, with all parameters written down, 
in a permanent place (e.g., a hard-backed book), while you are still doing the work, and to label 
all results with which experiment they belong to. Of course, if you will be working on computer, 
take back-ups seriously. Ideally, your report should contain enough information about the methods
 used to enable another student/researcher to replicate your experiments. The methods section should
  concisely describe your design as well as its materials and fabrication. Also specify which control
   strategy you are using for your robot and in case you are using a bioinspired design or another
    inspiration, state which animal(s)/character(s) it is inspired by and how. Lastly, describe the
     methodology of your human-robot interaction (HRI) experiment (procedure, measures/questions posed etc.).

\section{Results ($\sim\% 40$)}
The results section of your report should contain the actual results. No interpretations, 
no extrapolations, just the (processed) data. Displayed nicely with confidence limits included 
in some form — error bars, standard deviations, whatever. The description of how you processed
 your data should be included in either the methods or else the results section. Note that unexpected
  and outlying results should also be included, especially if they are repeatable. In theory, 
  negative results are just as important as positive results. It is just as scientifically valid 
  and useful to state that this approach to this problem doesn't work, as it is to find an approach
   that does work. As the project focuses on nonverbal communication in interactions with humans, 
   you should include results of one or more human-robot interaction (HRI) experiments evaluating
    this communication and demographic data for participants (age, gender, prior robot interaction experience).

\section{Discussion ($\sim\% 20$)}
The discussion section is where you put your interpretations, extrapolations etc. It normally 
starts with concrete conclusions that come directly from your data itself, then gradually becomes 
more abstract as you extrapolate into the future and into broader research areas. You should start 
concrete, to show people that you are working with fact and not fantasy. But there is no harm in 
claiming potential greater significance for your results, if justifiable. Go back and check again, 
that what you have written in your discussion section ties in properly with the results you included 
in your report. What is the main message that you want readers to take home? Make this the primary 
focus of your discussion! The rest of the discussion should provide supporting arguments for this, 
your main thread. Your discussion section could address, e.g., select aspects of the robot’s 
performance compared to the state-of-the-art and results and implications of your HRI experiment(s).



\begin{table}[h]
\caption{Typical Silicone-based soft materials}
\label{table_example}
\begin{center}
\begin{tabular}{ccccc}
\hline
&
\multicolumn{2}{c}{Ecoflex} 
& Elastosil & Sylgard\\
\hline
Type & 00-30 & 00-50 & M4601 & 184\\
Shore hardness & 00-30 & 00-50 & 28A & 50A\\
Elongation at break & 900\% & 980\% & 700\% & 150\%\\
\hline
\end{tabular}
\end{center}
\end{table}



   


\section{Conclusion ($\sim\% 5$)}

A conclusion may review the main points of the work, do not replicate the abstract 
as the conclusion. A conclusion might elaborate on the importance of the work or 
suggest applications, extensions, and improvements. 

\addtolength{\textheight}{-12cm}   % This command serves to balance the column lengths
 
%%%%%%%%%%%%%%%%%%%%%%%%%
\section*{Appendix}

Insert the link to an edited video (unlisted Youtube) of your project and a brief description 
of supplementary materials (code, properly labelled data, etc.). State each student’s contribution 
to the project and report according to the provided template on ItsLearning.



%%%%%%%%%%%%%%%%%%%%%%%%%



\begin{thebibliography}{99}

\bibitem{Wilson2007} J. F. Wilson, I. Norio, Bellows-type springs for robotics, 
Proc. Adv. Spring Technol. JSSE 60th Anniversary Int. Symp., 109–119 (2007).
 






\end{thebibliography}




\end{document}
